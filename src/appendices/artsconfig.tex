\chapter{ARTS Job Configuration Specification}
\label[app]{ap:artsconfig}

ARTS jobs are configured using a JSON document as a job specification. This section outlines the different configuration options for each of the components of an ARTS job. The basic outline of a job configuration can be seen below, the individual sections will be explained in greater depth in the following sections. A complete example can be seen at the end of this appendix.

\begin{lstlisting}[language=json]
{
    "base": { ... }
    "layers": [ ... ],
    "events": [ ... ],
    "handlers": [ ... ]
}
\end{lstlisting}

\section{Base}
The ``base'' attribute specifies the initial values which the layers and events will be applied to.

\subsection{File}
Read a newline delimited file(s) of tab separated unix timestamp, value pairs.

\textbf{Arguments:}
\begin{itemize}
\item \textbf{filepath:} a unix-style glob absolute filepath to the desired input files \\
\end{itemize}

\subsection{Value}
Use a specified value as the basis of the workload.

\textbf{Arguments:}
\begin{itemize}
\item \textbf{tnot:} The initial unix time where the workload will begin
\item \textbf{duration:} The duration of the generated workload (in seconds)
\item \textbf{interval:} How often a sample should be generated (in seconds)
\item \textbf{value:} The baseline value which should be applied for the duration of the workload
\end{itemize}

\section{Layers}
Layers form the core of ARTS workloads. Layers are applied as modifiers to the base which was described in the previous section.

\subsection{Sinusoid}
Sinusoids are used to represent normal patterns seen in traffic. The resulting sine wave is centered around positive one and applied as a multiplier to the existing workload. The sinusoid transform takes an amplitude between zero and one and a frequency specified in days as parameters.

\textbf{Arguments:}
\begin{itemize}
\item \textbf{amplitude:} Amplitude of the sine wave (between 0.0 and 1.0)
\item \textbf{frequency:} The frequency of the sinusoid (in seconds)
\item \textbf{offset:} Offset in time from the start of the workload (in seconds)
\end{itemize}

\subsection{Linear}
In order to represent steady growth (or decay) seen in the real-world traffic, we offer a linear transformation. The linear transformation takes a slope as a parameter and applies that slope to each point in the workload.

\textbf{Arguments:}
\begin{itemize}
\item \textbf{slope:} Amount of slope to apply to the input data
\end{itemize}

\subsection{Blur}
Real world traffic doesn't behave like a nice straight line, there are constant minor fluctuations in traffic patterns. The blur transformation takes the current workload and adds noise based on the Gaussian distribution. Using the current data point as the mean and a configurable standard deviation, a new value is generated and inserted back into the workload.

\textbf{Arguments:}
\begin{itemize}
\item \textbf{std\_dev:} The standard deviation which should be passed to the Gaussian distribution
\end{itemize}

\section{Events}
Events are modifiers applied to the workload over a discrete time. All events take a start and end time as parameters.

\subsection{Outage}
A period of time in which all generated data is zero. The outage event takes no extra arguments.

\subsection{Spike}
A short period of time when a large multiplier is applied to the base load.

\textbf{Arguments:}
\begin{itemize}
\item \textbf{peak:} The peak multiplier to be applied to the base workload.
\end{itemize}

\section{Handlers}
Every output tuple of ARTS can be passed to one or more handlers. Handlers can be used to store the generated data or feed it directly to another application.

\subsection{FileHandler}
The file handler outputs data to the specified file in a format which it can be read in again by the file base.

\textbf{Arguments:}
\begin{itemize}
\item \textbf{filename:} Name of the output file
\end{itemize}

\subsection{CRAFTSHandler}
This handler takes the data generated by ARTS and inserts it directly into the CRAFTS intermediate store.

\textbf{Arguments:}
\begin{itemize}
\item \textbf{url:} The url of the CouchDB instance which holds the target CRAFTS database
\item \textbf{db:} The name of the CRAFTS database within CouchDB
\end{itemize}

\section{Sample Job Configuration}
\begin{lstlisting}[language=json]
{
    "base": {
        "type": "file",
        "args": {"filename": "/data/wikipedia/*"}
    }
    "layers": [
        {
            "type": "linear",
            "args": {"slope": 5}
        },
        {
            "type": "fuzzing",
            "args": {"avg": 0.1, "dev": 0.05}
        }
    ],
    "events": [
        {
            "type": "outage",
            "args": {"start": 6, "end": 10}
        },
        {
            "type": "spike",
            "args": {"start": 20, "end": 25}
        }
    ],
    "handlers": [
        {
            "type": "FileHandler",
            "args": {"filename": "new_wiki.out"}
        }
    ]
}
\end{lstlisting}