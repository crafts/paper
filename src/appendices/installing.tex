\chapter{Running CRAFTS}
\label[app]{ap:installing}
This appendix outlines instructions on how to install, setup, and run an instance of CRAFTS. Further details of how to configure CRAFTS can be found in \Cref{ap:craftsconfig}.

\section{Installation}
CRAFTS supports a number of methods of installation. Below are detailed instructions for each of the deployment methods which CRAFTS offers. This guide assumes, unless stated otherwise, that a CouchDB instance is already accessible to the user.

\paragraph{Source.}
\texttt{Git} is required in order to acquire the CRAFTS source. To download the source, simply run:

\texttt{git clone git://github.com/crafts/crafts-core.git}

\noindent
This will create the \texttt{crafts-core} folder which contains all code necessary to run CRAFTS. 

\paragraph{Pip.} Installing with pip is quick and easy. Simply run:

\texttt{pip install crafts-core}

\paragraph{Docker.} Launching a Docker container requires a copy of the source code. Inside of the \texttt{crafts-core} directory is a \texttt{Dockerfile}, build the Docker container by running the following command:

\texttt{docker build -t crafts/crafts-core .}

\noindent
The Docker container comes with a CouchDB instance pre-installed and configured.

\paragraph{Vagrant.} Launching a VM through Vagrant requires a copy of the source code. Inside of the \texttt{crafts-core} directory is a \texttt{Vagrantfile}. From within the directory, run:

\texttt{vagrant up}

\noindent
to launch the CRAFTS VM. This VM includes a local pre-configured instance of CouchDB.

\section{Setup}
Setup is as simple as using the \texttt{crafts-cli} utility to configure a running CouchDB instance. Assuming the CouchDB instance is running locally, this can be done by running:

\texttt{crafts-cli init config.json}

\noindent
Further Information about the CRAFTS configuration file can be found in \Cref{ap:craftsconfig}. If CouchDB is installed remotely, run \texttt{crafts-cli help} to see the necessary flags.

\section{Running}
The primary executable for CRAFTS is the CRAFTS daemon, \texttt{craftsd}. Each of the installation methods above, except source, will add \texttt{craftsd} to the \texttt{PATH}. If CRAFTS was installed from source, \texttt{craftsd} can be found in the \texttt{crafts} directory beneath \texttt{crafts-core}.