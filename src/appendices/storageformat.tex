\chapter{Intermediate Storage Format}
\label[app]{ap:storage}

An example intermediate storage entry can be found at the end of this appendix. Below is a description of all fields which must be present in a valid sample document.
\begin{itemize}
\item \textbf{\_id:} Uniquely identifies the document within CouchDB. For sample documents, the id takes the form: ``\texttt\item \{role/timestamp/sample\}.''
\item \textbf{\_rev} The revision field. Used by CouchDB for consistency.
\item \textbf{timestamp:} The UTC time at which the sample was taken.
\item \textbf{role:} The role for which the sample was taken.
\item \textbf{hosts:} A list of all hosts within the role and a dictionary of the metrics recorded and their values.
\item \textbf{type:} Identifies the type of document. In this case it will always be ``sample.''
\end{itemize}

\begin{lstlisting}[language=json]
{
    "_id": "arts/2007-09-18T20:10:00/sample",
    "_rev": "1-74ec4360a516ae634edb61f840b12758",
    "timestamp": "2007-09-18T20:10:00",
    "role": "arts",
    "hosts": {
        "arts-1": {
            "requests": 3025
        }
    },
    "type": "sample"
}
\end{lstlisting}