Modern web services can see well over a billion requests per day. This sort of scale, as well as the advent of ``big data,'' has created a need for computational resources like never before. Data and services at such scale require advanced software and large amounts of computational resources to process requests in reasonable time. Advancements in cloud computing now allow us to acquire additional resources faster than ever before. We can scale systems up and down as required, allowing companies to meet the demand of their customers without having to purchase expensive hardware of their own. Unfortunately, these now routine scaling operations remain a primarily manual task. To solve this problem, we present CRAFTS (Cloud Resource Anticipation For Timing Scaling), a system for automatically identifying application throughput and predictively scaling cloud computing resources based on historical data. We also present ARTS (Automated Request Trace Simulator), a request based workload generation tool for constructing diverse and realistic request patterns for modern web applications. ARTS allows us to evaluate CRAFTS' algorithms on a wide range of scenarios. In this thesis, we outline the design and implementation of both ARTS and CRAFTS and evaluate the effectiveness of various prediction algorithms applied to real-world request data and artificial workloads generated by ARTS.